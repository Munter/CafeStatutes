\documentclass[a4paper,11pt,danish]{article}
\usepackage[T1]{fontenc}
\usepackage[utf8]{inputenc}
\usepackage{babel}

\title{Vedtægter for Foreningen til Cafédrift af 1993}
\author{}
\date{d. 10/3/2009}

\addtolength{\textheight}{100pt}
\topmargin=-50pt

\def\thesection{\S\arabic{section}}
\def\thesubsection{\textbf{Stk. \arabic{subsection}} }

\def\stk{
    \vspace{0.5em}
    \addtocounter{subsection}{1}
    \noindent\thesubsection \relax
}

\begin{document}

\maketitle

\section{}
\stk Foreningens navn er "Foreningen til Cafédrift af 1993". Foreningens hjemsted er Universitetsparken kbh. Ø.

\section{ }
\stk Foreningens formål er afholdelse af cafévirksomhed og andre arrangementer til fremme af socialt samvær på tværs af faggrupperne tilknyttet Det Naturvidenskabelige Fakultet ved Københavns Universitet.

\stk Kun foreningens medlemmer har adgang til foreningens arrangementer, herunder caféen. Foreningen skal fastsætte priserne så lavt som det er forretningsmæssigt forsvarligt.

\stk Ethvert overskud efter konsolidering skal uddeles som tilskud til studentersociale aktiviteter til gavn for de studerende ved Det Naturvidenskabelige Fakultet ved Københavns Universitet. Uddelingen forestås af foreningens bestyrelse.

\section{ }
\stk Som medlem af foreningen kan optages enhver studerende, ansat ved eller dimittend fra videregående uddannelsesinstitutioner.

\stk Bestyrelsen har desuden ret til at udstede særlige medlemskaber til folk der ikke opfylder kravene for medlemskab, sådanne medlemskaber er gyldige til næste bestyrelsesvalg. Denne type medlemmer har samme rettigheder som medlemmer optaget på normale vilkår. Generalforsamlingen informeres om de udstedte særlige medlemskaber og kan underkende bestyrelsens udstedelser hvis der er flertal for det.

\stk Kontingentet fastlægges hvert år på generalforsamlingen.

\section{ }
\stk Foreningen er ikke politisk.

\section{ }
\stk Foreningens øverste myndighed er generalforsamlingen. Ordinær generalforsamling afholdes årligt i marts måned.

\stk Indkaldelse til ordinær og ekstraordinær generalforsamling skal ske med mindst 2 ugers varsel og dagsorden skal offentliggøres senest 3 dage før generalforsamlingen. Offentliggørelse af indkaldelse og dagsorden skal ske via foreningens officielle hjemmeside og foreningens nyhedsmail

\stk Ekstraordinær generalforsamling kan indkaldes når et bestyrelsesmedlem eller 10 medlemmer ønsker det.

\stk Den ordinære generalforsamlings dagsorden skal mindst omfatte følgende punkter:

\begin{enumerate}
    \item Valg af dirigent og referent.
    \item Bestyrelsens beretning.
    \item Kassererens beretning.
    \item Godkendelse af regnskab.
    \item Orientering om uddeling af årets resultat.
    \item Fastsættelse af kontingent.
    \item Valg af revision og kritisk revision.
    \item Valg af bestyrelse.
    \item Indkomne forslag.
    \item Eventuelt.
\end{enumerate}

\stk På generalforsamlingen kan kun behandles sager, der står på dagsordenen

\stk Forslag til vedtægtsændringer skal være bestyrelsen i hænde senest 7 dage før generalforsamlingen.
    Forslag til punkter på dagsordenen skal ligeledes være bestyrelsen i hænde senest 7 dage før generalforsamlingen.

\stk Opstilling til bestyrelsesvalg skal ske skriftligt til bestyrelsen senest 7 dage før generalforsamlingen.

\stk Opstillingsberettiget er ethvert medlem af foreningen

\stk Bestyrelsen skal bestå af minimalt 3 og maksimalt 11 personer.

\stk Over halvdelen af de valgte bestyrelsesmedlemmer skal være fyldt 25 år (krav fra bevillingsmyndighederne).

\stk Procedure ved valg. Der udleveres stemmesedler med navne på samtlige kandidater. Ud for hver kandidat er afsat plads til et kryds. En gyldig stemme skal indeholde op til så mange krydser, som der er bestyrelsespladser på valg.

\stk Valget gennemføres herefter således:
    Først stemmes om de 6 første pladser til bestyrelsen blandt de opstillede, som er fyldt 25 år. Der kan sættes så mange krydser, som der er personer på valg, dog højst 6. De kandidater, der har flest stemmer vælges til bestyrelsen - dog skal der stemmes for hver enkelt kandidat af mindst halvdelen af de stemmeberettigede. Er ingen stemt for af halvdelen af de stemmeberettigede, fravælges den kandidat med færrest stemmer, og der foretages omvalg.
    Dernæst stemmes om resten af de opstillede - inklusive personer fravalgt i første runde - og der kan sættes op til så mange krydser, som der er personer på valg, men der kan dog kun sættes færre krydser end antallet af personer, der blev valgt i første runde. De personer, der har flest stemmer vælges til bestyrelsen - dog skal der stemmes for af mindst halvdelen af de stemmeberettigede.

\section{ }
\stk Afstemninger på generalforsamlingen afgøres ved simpelt flertal blandt de fremmødte.
    Undtaget er beslutninger omtalt i § 11 og § 12.
    Hvis ingen gør indsigelse derimod, foretages afstemninger ved håndsoprækning.

\stk Stemmeberettiget er ethvert medlem af foreningen ved personligt fremmøde.

\stk Hvis der er flere indbyrdes modstridende forslag til beslutning, stemmes om hvert af forslagene.
    Det forslag med færrest stemmer forkastes, og der stemmes om de resterende forslag, ved stemmelighed kan flere forslag forkastes.
    Når der kun er et forslag tilbage sættes dette til afstemning, hvor der stemmes for eller imod dette forslag.

\section{ }

\stk Foreningens daglige drift varetages af en bestyrelse, som nedsættes i overensstemmelse med de offentlige myndigheders krav. Bestyrelsen konstituerer sig selv med formand, næstformand og kasserer.

\stk Bestyrelsen er selvsupplerende blandt foreningens medlemmer mellem generalforsamlingerne, hvor en ny bestyrelse vælges.

\stk Alle medlemmer af foreningen har fuld møderet uden stemmeret til bestyrelsesmøderne.

\stk Bestyrelsen kan vælge at lukke møderne til enkelte særligt følsomme punkter - f.eks. personsager.

\section{}
\stk Bestyrelsen mødes ordinært mindst en gang om måneden. Formanden indkalder bestyrelsen senest 7 dage før mødet.

\stk Ethvert medlem af bestyrelsen kan indkalde til ekstraordinært bestyrelsesmøde med 3 dages varsel.

\stk Afstemninger på bestyrelsesmøderne foretages ved håndsoprækning og afgøres ved simpelt flertal blandt de fremmødte bestyrelsesmedlemmer.

\stk Bestyrelsen kan fastsætte yderligere regler for sit virke i en forretningsorden.

\section{}
\stk Foreningen tegnes ved underskrift to bestyrelsesmedlemmer.

\section{}
\stk Udmeldelse skal ske til bestyrelsen.

\stk Bestyrelsen kan ekskludere medlemmer, der groft overtræder foreningens ordensregler eller i øvrigt overtræder gældende lovgivning, såfremt det har relevans for foreningen.

\section{}
\stk Vedtægtsændringer foretages på generalforsamlingen og kræver kvalificeret flertal (2/3 af de afgivne stemmer).

\stk Vedtægtsændringer træder i kraft efter den generalforsamling de bliver vedtaget på.

\section{}
\stk Foreningen kan opløses ved at der på 2 på hinanden følgende generalforsamlinger er kvalificeret flertal (2/3 af de afgivne stemmer) herfor.

\stk Ved den sidste generalforsamling vedtages, hvad der skal ske med foreningens midler.

\section{}
\stk Foreningen hæfter kun med sin egen kapital, der ved stiftelsen udgør kr. 45.000.

\section{}
\stk Foreningens regnskabsår løber fra 1. januar til 31. december.

\stk Regnskabet skal før den ordinære generalforsamling revideres af to blandt medlemmerne udenfor bestyrelsen valgte revisorer.


\end{document}