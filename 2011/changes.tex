\documentclass[a4paper,12pt,danish]{article}
\usepackage[T1]{fontenc}
\usepackage[utf8]{inputenc}
\usepackage{babel}

\title{Forslag til ændringer af vedtægter for Foreningen til Cafédrift af 1993}
\author{Bestyrelsen}
\date{d. 23/3/2011}

\newcommand\cit[1]{
    \begin{quote}
        \textit{``#1''}
    \end{quote}
}

\newcommand\who[1]{
    \textbf{Foreslagsstiller:} #1\\
}

\newcommand\why[1]{
    \textbf{Motivation:} #1\\
}

\def\thesection{Ændringsforslag \arabic{section}}
\newcommand\change[1]{
    \section{}
    #1
}
\def\thesubsection{Alternativ \Alph{subsection}}
\newcommand\alt[1]{
    \subsection{}
    #1
}

\begin{document}

\maketitle


\change{
    \S1 Stk. 1:
    \cit{Foreningens navn er ``Foreningen til Cafédrift af 1993''. Foreningens hjemsted er Universitetsparken kbh. Ø.}
    Ændres til:
    \cit{Foreningens navn er ``Foreningen til Cafédrift af 1993'', i daglig tale ``Caféen?''. Foreningens hjemsted er:\\
    Universitetsparken 13A\\
    2100 København Ø.}
    \who{Christian Stokholm, Thomas N. Barnholdt, Peter Müller}
    \why{Vedtægten bør afspejle realiteten}
}

\change{
    \S3 Stk. 1:
    \cit{Som medlem af foreningen kan optages enhver studerende, ansat ved eller dimittend fra videregående uddannelsesinstitutioner.}
    Ændres til:
    \cit{Som medlem af foreningen kan optages enhver studerende ved, ansat ved eller dimittend fra en videregående uddannelsesinstitution.}
    \who{Christian Stokholm, Thomas N. Barnholdt, Peter Müller}
    \why{Redaktionel ændring}
}

\change{
	Efter \S3 Stk. 1 tilføjes nyt stykke:
    \cit{Medlemsskabet er af et års varighed, og følger universitets studieår.}
    \who{Christian Stokholm, Thomas N. Barnholdt, Peter Müller}
    \why{Vedtægterne bør afspejle organisationens praksis}
}

\change{
	Efter \S10 Stk. 2 tilføjes nyt stykke:
    \cit{Af bestyrelsen ekskluderede medlemmer har ret til at få deres sag bragt op ved generalforsamling, og bestyrelsen skal informere det ekskluderede medlem herom.}
    \who{Peter Müller}
    \why{Ekskluderede medlemmer bør have mulighed for at få deres sag bragt op til revurdering af generalforsamlingen hvis medlemmet føler sig uretfærdigt behandlet.}
}

\change{
	Efter \S10 tilføjes nyt stykke:
    \cit{Generalforsamlingen informeres om antallet af ekskluderede medlemmer, og kan på forespørgsel gå ind i personsager. Generalforsamlingen kan underkende bestyrelsens ekskludering af medlemmer.}
    \who{Peter Müller}
    \why{Generalforsamlingen bør være informeret om eksklusioner og have mulighed for at gå ind i sagen og revurdere bestyrelsens beslutning.}
}

\change{
	Til \S8 Stk. 1 tilføjes ordlyden:
    \cit{Indkaldelse og dagsorden offentliggøres inden mødet på foreningens hjemmeside.}
    \who{Peter Müller}
    \why{Bestyrelsens arbejde bør være mere synligt, og det bør være lettere at få indblik i Caféen?s forretningsgange som menigt medlem.}
}

\change{
	\S5 Stk. 2:
    \cit{Offentliggørelse af indkaldelse og dagsorden skal ske via foreningens officielle hjemmeside og foreningens nyhedsmail}
    Ændres til:
    \cit{Offentliggørelse af indkaldelse samt dagsorden skal ske via foreningens officielle hjemmeside, foreningens nyhedsmail og synligt opslag i foreningens lokaler.}
    \who{Christian Stokholm, Thomas N. Barnholdt, Peter Müller}
    \why{Foreningen bør tilstræbe at dække så bredt et publikum som muligt med annonceringen af generalforsamlinger.}
}

\change{
	\S10 Stk. 1:
    \cit{Foreningen tegnes ved underskrift to bestyrelsesmedlemmer.}
    Ændres til:
    \cit{Foreningen tegnes af formand og kasserer, efter forudgående godkendelse af flertallet af bestyrelsen.}
    \who{Christian Stokholm}
    \why{2 tilfældige bestyrelsesmedlemmer bør ikke kunne indgå økonomisk bindende kontrakter for foreningen.}
}

\change{
	Ordlyden i \S12 Stk. 1:
    \cit{2 på hinanden følgende generalforsamlinger}
    Ændres til:
    \cit{2 på hinanden følgende ordinære generalforsamlinger}
    \who{Christian Stokholm, Thomas N. Barnholdt, Peter Müller}
    \why{Kupsikring. Foreningen har nået en størrelse organisatorisk og økonomisk, hvor det er nødvendigt at sikre sig mod kupforsøg.}
}

\change{
	\S13 Stk. 1:
    \cit{Foreningen hæfter kun med sin egen kapital, der ved stiftelsen udgør kr. 45.000.}
    Ændres til:
    \cit{Foreningen hæfter kun for sine forpligtelser med den til enhver tid tilhørende formue.}
    \who{Christian Stokholm, Thomas N. Barnholdt, Peter Müller}
    \why{Foreningens formue overstiger de oprindeligt indskudte 45.000 kroner. Derudover bør en så lille formue ikke udstilles i vedtægterne, da det kan give problemer med at indgå kontrater eller handler for større beløb}
}

\change{
	Efter \S12 Stk. 1 tilføjes nyt Stk 2:
    \cit{Bestyrelse og medlemmer hæfter ikke personligt for en evt. gæld i foreningen.}
    \who{Christian Stokholm, Thomas N. Barnholdt, Peter Müller}
    \why{Økonomisk sikring af bestyrelse og medlemmer}
}

\change{
	Efter \S14 Stk. 2 tilføjes nyt Stk. 3:
    \cit{Regnskabet udsendes sammen med dagsordenen 3 dage inden generalforsamlingen.}
    \who{Christian Stokholm, Thomas N. Barnholdt, Peter Müller}
    \why{Det bør være muligt for medlemmer at kunne forberede sig på at diskutere foreningens økonomiske strategi fremfor at skulle koncentrere sig om at forstå regnskabet ved prima vista på generalforsamlingen.}
}

\change{
	\S10 Stk. 1 fjernes
    \who{Christian Stokholm, Thomas N. Barnholdt, Peter Müller}
    \why{Medlemsskab ophører efter et år og indebærer ingen forpligtelser af det enkelte medlem. Om et medlem udmelder sig har ingen praktisk konsekvens for bestyrelsen.}
}

\change{
	Generel redaktionel ændring af vedtægter. Enkelte stykker flyttes rundt således at de enkelte paragraffer bliver tematiserede under følgende overskrifter:
    \begin{itemize}
        \item \S1: Navn og hjemsted
        \item \S2: Formål og virke
        \item \S3: Medlemsskab og kontingent
        \item \S4: Politiske tilhørsforhold
        \item \S5: Daglig ledelse
        \item \S6: Bestyrelsens virke
        \item \S7: Generalforsamling
        \item \S8: Stemmeret
        \item \S9: Vedtægtsændringer
        \item \S10: Tegningsret
        \item \S11: Opløsning
        \item \S12: Hæftelse
        \item \S13: Regnskab og revision
    \end{itemize}
    Generalforsamlingen giver bestyrelsen mandat til at lave redaktionelle ændringer i vedtægterne som ikke har nogen reel indflydelse på organisationens struktur eller politik, således at disse bliver lettere at forstå.\\

    \who{Christian Stokholm, Thomas N. Barnholdt, Peter Müller}
    \why{De eksisterende vedtægter har intet tematisk overblik, og enkelte stykker under nogle paragraffer hører i virkeligheden til under andre temaer. Derudover kan en ændret rækkefølge som ovenstående føre til en mere konsekvent opbygning med færre forklarende paranteser. Se eventuelt forumtråden med et samlet forslag, som dog også inkludere enkelte øvrige ændringsforslag:\\
    http://forum.cafeen.org/index.php?s=\&showtopic=5864\&view=findpost\&p=218996}
}

\change{
	\S3 Stk. 4:
Forslag til ny stykke.
    \cit{Det er bestyrelsens ret at udstede særlige livsvarige medlemskaber, til enhver som kan fremvise tattovering af foreningens logo på egen krop. Sådanne medlemskaber skal som alle andre kunne eftertjekkes, hvorfor indehavers tattovering skal kunne fremvises på lige fod med ordinære medlemskort. Denne type medlemmer har samme rettigheder som medlemmer optaget på normale vilkår. Generalforsamlingen informeres om de udstedte særlige medlemskaber og kan underkende bestyrelsens udstedelser hvis der er flertal for det. Såfremt tattoveringen ikke kan fremvises, eller kan konstateres fjernet, bortfalder medlemskabet.}
    \who{Thomas T. Lund}
    \why{Det er muligt at der i fremtiden vil opstå situationer hvor et livsvarigt langt medlemsskab vil være på sin plads at tildele.}
}

\change{
	\S7 Stk. 1:
    \cit{Foreningens daglige drift varetages af en bestyrelse, som nedsættes i overensstemmelse med de offentlige myndigheders krav. Bestyrelsen konstituerer sig selv med formand, næstformand og kasserer.}
    Ændres til:
    \cit{Foreningens daglige drift varetages af en bestyrelse, som nedsættes i overensstemmelse med de offentlige myndigheders krav. Bestyrelsen konstituerer sig selv med formand og kasserer.}
    \who{Jenny-Margrethe Vej}
    \why{Det er ikke lovpligtigt, at næstformanden SKAL vælges ind i bestyrelsen. Det er KUN lovpligtigt mht. formanden og kasseren, hvorfor jeg mener, vi bør ændre dette til, at bestyrelsen kun konstituerer sig selv med formand og kasserer, således næstformanden kan indsuppleres i fremtiden.}
}

\change{
	\S5 Stk. 12:
    \cit{Valget gennemføres herefter således:\\
Først stemmes om de 6 første pladser til bestyrelsen blandt de opstillede, som er fyldt 25 år. Der kan sættes så mange krydser, som der er personer på valg, dog højst 6. De kandidater, der har flest stemmer vælges til bestyrelsen - dog skal der stemmes for hver enkelt kandidat af mindst halvdelen af de stemmeberettigede. Er ingen stemt for af halvdelen af de stemmeberettigede, fravælges den kandidat med færrest stemmer, og der foretages omvalg.\\
Dernæst stemmes om resten af de opstillede - inklusive personer fravalgt i første runde - og der kan sættes op til så mange krydser, som der er personer på valg, men der kan dog kun sættes færre krydser end antallet af personer, der blev valgt i første runde. De personer, der har flest stemmer vælges til bestyrelsen - dog skal der stemmes for af mindst halvdelen af de stemmeberettigede.}
    Ændres til:

    \alt{
        \S5 Stk. 12 erstattes med:
        \cit{
            Valget gennemføres herefter således:\\
            Valget foregår over 2 runder. En runde betragtes enten som et tillidsvalg eller som et kampvalg.\\
            Hvis antallet af opstillede i en runde overstiger antallet af pladser på valg i denne runde, betragtes runden som et kampvalg.\\
            Hvis antallet af opstillede i en runde er mindre eller det samme som antallet af pladser på valg i denne runde, betragtes runden som et tillidsvalg.\\
            \\
            1. Runde:\\
            I første runde stemmes om de opstillede som er fyldt 25 år. I denne runde er der op til 6 pladser på valg, dog højest lige så mange pladser som der er opstillede over 25 år.\\
            Der kan i første runde stemmes på op til så mange som er opstillede i denne runde.\\
            Betragtes første runde som et kampvalg, vælges de 6 opstillede som har fået flest stemmer, såfremt de pågældende har fået stemmer fra mindst halvdelen af de stemmeberettigede. De opstillede i denne runde som ikke vælges, opstilles sammen med de opstillede under 25 år i anden runde.\\
            I tilfælde af stemmelighed blandt to eller flere, således at der opstår tvivl om hvem der vælges, foretages kampvalg om de pågældende. Ved dette kampvalg kan der stemmes på op til 1 mindre end de opstillede ved dette valg. Såfremt der igen opnås stemmelighed blandt mindre end antal opstillede ved dette valg, fjernes den med færrest stemmer, og der foretages genvalg blandt de resterende efter samme procedure. Såfremt der opnås stemmelighed mellem alle opstillede ved dette valg, opstilles de pågældende sammen med de opstillede under 25 år i anden runde.\\
            Betragtes første runde som tillidsvalg, vælges de op til 6 opstillede som har fået stemmer fra mindst halvdelen af de stemmeberettigede. Hvis en eller flere ved tillidsvalg ikke opnår stemmer fra mindst halvdelen af de stemmeberettigede, betragtes det som implicit mistillidsvotum fra generalforsamlingen, hvorved pågældende ikke opstilles til anden runde.\\
            \\
            2. Runde:\\
            I anden runde stemmes om resten af de opstillede, inklusive personer fravalgt i første runde uden at have opnået implicit mistillidsvotum. Der kan i anden runde vælges 1 mindre end der blev valgt i første runde.\\
            Der kan i anden runde stemmes på op til så mange som der er opstillede i anden runde.\\
            Betragtes anden runde som et kampvalg, vælges så mange som der er bestyrelsespladser på valg, ud fra hvem der har fået flest stemmer, såfremt de pågældende har fået stemmer fra mindst halvdelen af de stemmeberettigede.\\
            I tilfælde af stemmelighed blandt to eller flere således at der opstår tvivl om hvem der vælges, foretages kampvalg om de pågældende. Ved dette kampvalg kan der stemmes på op til 1 mindre end antallet af de opstillede ved dette valg. Såfremt der igen opnås stemmelighed blandt to eller flere, fjernes den med færrest stemmer, og der foretages genvalg blandt de resterende efter samme procedure. Såfremt der opnås stemmelighed mellem alle opstillede ved dette valg, vælges ingen af de pågældende.\\
            Betragtes anden runde som tillidsvalg, vælges så mange, som der er bestyrelsespladser på valg som har fået stemmer fra mindst halvdelen af de stemmeberettigede. Hvis en eller flere ved tillidsvalg ikke opnår stemmer fra mindst halvdelen af de stemmeberettigede, betragtes det som implicit mistillidsvotum fra generalforsamlingen, hvorved pågældende ikke vælges til bestyrelsen.
        }
        \who{Thomas T. Lund}
        \why{Der mangler mere klare retningslinier omkring afstemningen, når der stemmes på den nye bestyrrelse}
    }

    \alt{
        \S5 Stk. 12 erstattes med:
        \cit{
            Valget forgår over to runder, der hver betragtes enten som et kampvalg eller et tillidsvalg.\\
            En runde betragtes som et kampvalg, hvis antallet af opstillede kandidater overstiger antallet af bestyrelsespladser på valg i den pågældende runde – ellers betragtes runden som et tillidsvalg.\\
            Der kan i begge runder stemmes på op til så mange, som der er opstillede.\\
            \\
            1. runde:\\
            \\
            I første runde stemmes om de opstillede som er fyldt 25 år. I denne runde er der op til 6 bestyrelsespladser på valg, dog højest antallet af opstillede over 25 år.\\
            \\
            Betragtes første runde som et kampvalg, vælges de 6 opstillede som har fået flest stemmer, såfremt de pågældende har fået stemmer fra mindst halvdelen af de stemmeberettigede. Stemmelighed er behandlet i stk. 13.\\
            \\
            Betragtes første runde som tillidsvalg, vælges de opstillede, som har fået stemmer fra mindst halvdelen af de stemmeberettigede. Hvis en eller flere ved tillidsvalg ikke opnår stemmer fra mindst halvdelen af de stemmeberettigede, betragtes det som implicit mistillidsvotum fra generalforsamlingen.\\
            \\
            Alle kandidater, som ikke vælges i første runde og som ikke har fået implicit mistillidsvotum, genopstilles i anden runde.\\
            \\
            2. runde:\\
            \\
            I anden runde stemmes om de opstillede under 25 år samt de genopstillede fra første runde. I anden runde er antallet af bestyrelsespladser på valg 1 mindre and antallet af valgte i første runde.\\
            \\
            Betragtes anden runde som et kampvalg, vælges – ud fra stemmetal – så mange som der bestyrelsespladser på valg, såfremt de pågældende har fået stemmer fra mindst halvdelen af de stemmeberettigede. Stemmelighed er behandlet i Stk. 13.\\
            \\
            Betragtes anden runde som tillidsvalg, vælges de opstillede, som har fået stemmer fra mindst halvdelen af de stemmeberettigede. Hvis en eller flere ved tillidsvalg ikke opnår stemmer fra mindst halvdelen af de stemmeberettigede, betragtes det som implicit mistillidsvotum fra generalforsamlingen, hvorved pågældende ikke vælges til bestyrelsen.
        }
        Efter \S5 Stk. 12 tilføjes nyt Stk. 13:
        \cit{
            I tilfælde af stemmelighed blandt to eller flere, således at der opstår tvivl om hvem der vælges, foretages kampvalg om de pågældende. Ved dette valg kan der stemmes på op til 1 mindre end antallet af opstillede ved dette valg. Såfremt der ved dette valg opnås stemmelighed blandt færre end antal opstillede ved dette valg, fjernes den med færrest stemmer (hvis en sådan findes), og der foretages genvalg blandt de resterende efter samme procedure. Såfremt der opnås stemmelighed mellem alle opstillede ved dette valg, vælges ingen af de pågældende i den aktuelle runde.
        }
        \who{Toke Nørbjerg}
        \why{Forsøg på at simplificere Ændringsforslag \arabic{section}}
    }
}

\end{document}
